\documentclass[a4paper,10pt]{article}
\usepackage[utf8]{inputenc}
\usepackage[french]{babel}
\usepackage[T1]{fontenc}
\usepackage{lmodern}
\usepackage{graphicx}
\usepackage{listings}
\usepackage{xcolor}
\usepackage{textcomp}
\setlength{\textwidth}{16cm}
\setlength{\marginparwidth}{0cm}
\setlength{\oddsidemargin}{0cm}
\setlength{\headheight}{0cm}
\setlength{\topmargin}{0cm}
\setlength{\headsep}{0cm}
\setlength{\textheight}{25cm}
\setlength{\footskip}{0cm}
\setlength{\marginparsep}{0cm}
\lstset{basicstyle=\small\ttfamily,breaklines=true}

%opening
\title{Compte Rendu TP NACHOS}
\author{EUDES Robin, ROSSI Ombeline, BADAMO Romain, MORISON Jack}

\begin{document}
\maketitle
\tableofcontents
\newpage
\section{Etape 2}
\subsection{Partie I : Introduction}
A l'exécution de putchar.c, on s'attends à avoir le retour suivant :
\begin{verbatim}
 $ ./nachos -x ./putchar
 abcd
 [...] # retour du syscall halt
\end{verbatim}
\subsection{Partie II : Entrées-sorties asynchrones}
\subsubsection{Observation de progtest.cc}
\begin{verbatim}
 ./nachos-userprog -c
test
test # retour de la console
arret si on tape sur q et rien ensuite
arret si on tape sur qMachine halting! # retour de la console
[...]
\end{verbatim}
La console se ferme sur la lecture du caractère ``q'' , et ignore tout ce qu'on a put saisir ensuite.

\subsubsection{ Modification de progtest.cc}
\begin{verbatim}
#ifdef CHANGED
  if (ch == 'q'|| ch ==EOF || ch=='\0')
  return;
#endif
\end{verbatim}
Cette modification sur le test d'arrêt de la console suffit à prendre en compte la terminaison de l’entrée correc-
tement : fin de fichier ou, sur un tty, \^D .

\end{document}
